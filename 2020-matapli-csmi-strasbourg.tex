\documentclass{matapli}

\usepackage[utf8]{inputenc}
\usepackage[T1]{fontenc}

\begin{document}

\titre[Master CSMI]{L'UE projet au master CSMI}

\communication{Prud'homme Christophe, Caldini-Queiros Céline}{Université de Strasbourg}

\begin{soustitre}
Les étudiants du master CSMI en projet en entreprise.
\end{soustitre}

\section*{Master CSMI}
Le cursus CSMI forme des étudiants à des compétences clés en mathématiques appliquées telles que la science de la donnée, la modélisation, simulation optimisation et calcul haute performance.
Le but est que ces étudiants obtiennent la capacité et les clés de s'adapter aux exigences et évolutions rapides de la révolution du numérique.
Les étudiants sont amenés à travailler en R\& D d'entreprise, ou en secteur public, avec une poursuite en doctorat souvent choisie. Les diplômés trouvent un emploi très peu après voir avant l'obtention du diplôme.
Bien que basé sur la théorie, les cours s'organisent autours de projets, qui permettent la mise en application des principes appris.
Un séminaire quasi-hebdomadaire permet une rencontre entre les étudiants et les entreprises du territoire alsacien, qui viennent présenter l'utilisation des mathématiques et les possibilités d'intégrations des étudiants en leurs sein.
\section*{La collaboration avec les entreprises : projet intégrateurs}
Le but des projets intégrateurs est la mise en place des acquis mathématiques dans des conditions "réelles".
Les étudiants, à l'aide d'un encadrant académique et un encadrant externes, sont intégrés à un projet pluridisciplinaire ou en collaboration avec une entreprise.
Bien que le projet soit à visée pédagogique et s'intègre intégralement dans le cursus de l'étudiant (l'encadrant académique en est le garant), il permet une immersion de l'étudiant dans le monde de l'entreprise, une première découverte des codes qui y sont associé et une réalisation que les compétences acquises dans le master ont en effet une application dans sa vie professionnelle future.

L'UE est présente aux semestre 2 et 3, accompagnée par un coordinateur pendant 26 heures, mais avec une journée par semaine dédiée à la réalisation du projet.
Une semaine est réservée fin mai pour les M1 et fin janvier pour les M2 afin de finaliser les projets et préparer la soutenance, qui pour les M1 est un exercice nouveau et difficile.
\subsection*{Rapport et soutenance}
L'écriture d'un rapport complet, comprenant les enjeux et le contexte du projet, ainsi que les bases scientifiques à actionner pour le bon déroulement ainsi que la description des choix de gestion de projet faits par l'étudiant, est un exercice fastidieux et souvent nouveau pour les étudiants. C'est pourquoi c'est un point important de l'évaluation de l'UE.

Une soutenance devant les personnels de l'entreprise, parfois à huis clos mais majoritairement devant l'intégralité de la promotion est également un exercice nouveau et qui nécessite une préparation suffisante.

L'importance de ces deux aspects de l'UE projet est mise en avant et permet de préparer les étudiants à leurs futures responsabilités en entreprise.

\subsection*{Intérêt pédagogique et professionnel}
Les projets sont une véritable opportunité pour les étudiants de découvrir des méthodes mathématiques qui ne sont pas forcément abordées, ou qui ne sont pas détaillées, dans leur cursus. Il est également une occasion de découvrir comment mener à bien un travail, en groupe ou individuellement.

Les projets sont également un véritable tremplin vers l'emploi. En M1, ils permettent de préparer le stage de fin de première année. En M2, ils permettent non seulement de préparer le stage de fin d'études mais peuvent conforter les entreprises vers une éventuelle embauche.

\section*{Cemosis et CSMI}
La proximité de la structure cemosis (CEntre de MOdélisation et de SImulation de Strasbourg) ainsi que le réseau professionnel des membres de la structures, permet d'obtenir une sélection intéressante de projets à présenter aux étudiants. 

La structure cemosis s'intègre à l'écosystème économique local à l'aide entre autres de sa participation aux différents réseaux d'innovation organisés par la CCI Alsace-Grand est. Cela permet de se faire connaitre au niveau local et de maintenir un lien de qualité avec les entreprises, en réalisant des projets de collaborations scientifiques.

La plateforme cemosis s'appuie sur le master CSMI ainsi que l'UE projet pour lancer des collaborations scientifiques qui permettront à terme aux étudiants de créer un réseau d'anciens. En effet, une fois embauchés les étudiants pourront eux aussi proposer des sujet aux nouvelles promotions.
\end{document}
